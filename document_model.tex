\documentclass[UTF8,12px,a4paper]{ctexart} % use larger type; default would be 10pt
%a4paper: width = 17cm
\usepackage{comment} % comment for more than one line

\begin{comment}
This is a TeX file using some common-used packages & skills in only one simple script,
and as a easy-used model for Chinese writer using CTeX-kit.
I uploaded the PDF file generated by this TeX script as well.
\end{comment}

%\usepackage[utf8]{inputenc}set input encoding (not needed with XeLaTeX)

%%% PAGE DIMENSIONS
\usepackage[margin=1in]{geometry} % to change the page dimensions
%\geometry{a4paper} or letterpaper (US) or a5paper or....
% \geometry{margin=2in} % for example, change the margins to 2 inches all round
% \geometry{landscape} % set up the page for landscape
%   read geometry.pdf for detailed page layout information

\usepackage{graphicx} % support the \includegraphics command and options
\graphicspath{{fig/}} % storage figure in a sub-folder
% \usepackage[parfill]{parskip} % Activate to begin paragraphs with an empty line rather than an indent

%%% PACKAGES
\usepackage{abstract} % for abstract
\usepackage{amsmath}
\usepackage{booktabs} % for much better looking tables
\usepackage{cite} % reference
\usepackage{array} % for better arrays (eg matrices) in maths
\usepackage{paralist} % very flexible & customisable lists (eg. enumerate/itemize, etc.)
\usepackage{verbatim} % adds environment for commenting out blocks of text & for better verbatim
\usepackage{subfig} % make it possible to include more than one captioned figure/table in a single float
% These packages are all incorporated in the memoir class to one degree or another...
\usepackage{cases} %equation set
\usepackage{multirow} %use table
\usepackage{lastpage} %get the total page
\usepackage{algorithm}
\usepackage{algorithmic}
\usepackage{comment} % comment for more than one line
\hypersetup{colorlinks,linkcolor=black,anchorcolor=black,citecolor=black,
pdfstartview=FitH,bookmarksnumbered=true,bookmarksopen=true,} % set href in tex & pdf

%%% HEADERS & FOOTERS
\usepackage{fancyhdr} % This should be set AFTER setting up the page geometry
\pagestyle{fancy} % options: empty , plain , fancy
\fancyhf{}
%in xelatex use:
\lhead{}\chead{学号:XXX \quad 姓名:YYY}\rhead{}
\lfoot{}\cfoot{第 \thepage 页,共 \pageref{LastPage} 页}\rfoot{}
\renewcommand{\headrulewidth}{0.5pt} % customise the layout...
\renewcommand{\footrulewidth}{0.5pt}

%%% SECTION TITLE APPEARANCE
\usepackage{sectsty}
\allsectionsfont{\sffamily\mdseries\upshape} % (See the fntguide.pdf for font help)
% (This matches ConTeXt defaults)
\usepackage{indentfirst}
\setlength{\parindent}{2em}
\numberwithin{equation}{section} %formed the equation number as (1.1)...

%%% ToC (table of contents) APPEARANCE
\usepackage[notlot,notlof]{tocbibind} % Put the bibliography in the ToC
\usepackage[titles,subfigure]{tocloft} % Alter the style of the Table of Contents
%\renewcommand{\cftsecfont}{\rmfamily\mdseries\upshape}
%\renewcommand{\cftsecpagefont}{\rmfamily\mdseries\upshape} % No bold!

%\renewcommend\tableofcontents{\heiti 目录}
%\renewcommend\refname{\heiti 参考文献}
\renewcommand{\contentsname}{\heiti 目录}
\renewcommand{\refname}{\heiti 参考文献}
%%% END Article customizations

%%% The "real" document content comes below...

\title{MYTITLE}
\author{ME}
%\date{} Activate to display a given date or no date (if empty),
         % otherwise the current date is printed

\begin{document}
\maketitle
\thispagestyle{fancy} % set the page style of the first page
{\noindent \Large \heiti 摘要}\\
% Chinese abstract
\par
{\noindent \Large \textbf{Abstract}}\\
% English abstract
\par
{\noindent \Large \heiti 关键词}\\
% Chinese keywords
\par
{\noindent \Large \textbf{Keywords}}\\
% English keywords
\tableofcontents
\section{\heiti 尺寸计算}
\subsection{\heiti 齿轮尺寸}
通常用分度圆\emph{circle}上的螺旋角$\beta$进行\underline{几何尺寸}的计算\cite{wang},一般有$\beta=8 \sim 20^{\circ}$,尺寸计算如表\ref{tab1}\\
\begin{table}[!hbp]
\centering
\begin{tabular}{c|c}
	\hline
	名称 & 公式\\
	\hline
	分度圆直径 & $d=m_n\times z/\cos \beta$\\
	齿顶圆直径 & $d_a=m_n\times (z/\cos \beta+2)$\\
	齿根圆直径 & $d_f=d_a-4.5\times m_n$\\
	\hline
\end{tabular}
\caption{齿轮参数的计算}
\label{tab1}
\end{table}
利用单圆弧绘法,即三点圆弧近似的方法计算齿轮参数\\
\begin{figure}[htb]
	\includegraphics[width=15cm]{cilun.png}
	\caption{齿轮计算方法示意图}
	\label{fig1}
\end{figure}
在图\ref{fig1}中以齿轮轴孔为中心建立坐标系XOY,设圆周方程为:\\
\begin{equation}
	(x-x_c)^2+(y-y_c)^2=r_c^2
\end{equation}\\
计算圆周上三个特定点:1,2,3的坐标:\\
\begin{equation}
	\begin{cases}
	x_1=-0.5d_a\sin(0.698/z)\\
	y_1=-0.5d_a\cos(0.698/z)
	\end{cases}
\end{equation}\\
\begin{equation}
	\begin{cases}
	x_2=-0.5d_f\sin(2.356/z)\\
	y_2=-0.5d_f\cos(2.356/z)
	\end{cases}
\end{equation}\\
\begin{equation}
	\begin{cases}
	x_3=-0.5d_o\sin(1.57/z)\\
	y_3=-0.5d_o\cos(1.57/z)
	\end{cases}
\end{equation}\\
$$
	\text{while} \quad d_o=(1.25d_a+d_f)/4.5
$$
证毕。

\section{\heiti 绘图结果}
算法\ref{alg1}
\begin{algorithm}
\caption{Calculate $y = x^n$}
\label{alg1}
\begin{algorithmic}[1] %这个1 表示每一行都显示数字
\REQUIRE $n \geq 0 \vee x \neq 0$
\ENSURE $y = x^n$
\STATE $y \Leftarrow 1$
\IF{$n < 0$}
\STATE $X \Leftarrow 1 / x$
\STATE $N \Leftarrow -n$
\ELSE
\STATE $X \Leftarrow x$
\STATE $N \Leftarrow n$
\ENDIF
\WHILE{$N \neq 0$}
\IF{$N$ is even}
\STATE $X \Leftarrow X \times X$
\STATE $N \Leftarrow N / 2$
\ELSE[$N$ is odd]
\STATE $y \Leftarrow y \times X$
\STATE $N \Leftarrow N - 1$
\ENDIF
\ENDWHILE
\end{algorithmic}
\end{algorithm}
又有\ref{alg2}
    \begin{algorithm}[t]
    \caption{ Framework of ensemble learning for our system.}
    \label{alg2}
    \begin{algorithmic}
    \STATE {\bf Input:} \\ %算法的输入参数:Input
    The set of positive samples for current batch, $P_n$;\\
    The set of unlabelled samples for current batch, $U_n$;\\
    Ensemble of classifiers on former batches, $E_{n-1}$;
    \STATE {\bf Output:} \\ %算法的输出:Output
    Ensemble of classifiers on the current batch, $E_n$;
    \STATE Extracting the set of reliable negative and/or positive samples $T_n$ from $U_n$ with help of $P_n$;
    \label{ code:fram:extract }%对此行的标记,方便在文中引用算法的某个步骤
    \STATE Training ensemble of classifiers $E$ on $T_n \cup P_n$, with help of data in former batches;
    \label{code:fram:trainbase}
    \STATE $E_n=E_{n-1}\cup E$;
    \label{code:fram:add}
    \STATE Classifying samples in $U_n-T_n$ by $E_n$;
    \label{code:fram:classify}
    \STATE Deleting some weak classifiers in $E_n$ so as to keep the capacity of $E_n$;
    \label{code:fram:select}
    \RETURN $E_n$; %算法的返回值
    \end{algorithmic}
    \end{algorithm}
以及\ref{alg3}
    \begin{algorithm}[t]
    \caption{An example for format For \& While Loop in Algorithm}
	\label{alg3}
    \begin{algorithmic}[1]
    \FOR{each $i \in [1,9]$}
    \STATE initialize a tree $T_{i}$ with only a leaf (the root);\
    \STATE $T=T \cup T_{i};$\
    \ENDFOR
    \FORALL {$c$ such that $c \in RecentMBatch(E_{n-1})$}
    \label{code:TrainBase:getc}
    \STATE $T=T \cup PosSample(c)$;
    \label{code:TrainBase:pos}
    \ENDFOR
    \FOR{$i=1$; $i<n$; $i++$ }
    \STATE $//$ Your source here;
    \ENDFOR
    \FOR{$i=1$ to $n$}
    \STATE $//$ Your source here;
    \ENDFOR
    \STATE $//$ Reusing recent base classifiers.
    \label{code:recentStart}
    \WHILE {$(|E_n| \leq L_1 )and( D \neq \phi)$}
    \STATE Selecting the most recent classifier $c_i$ from $D$;
    \STATE $D=D-c_i$;
    \STATE $E_n=E_n+c_i$;
    \ENDWHILE
    \label{code:recentEnd}
    \end{algorithmic}
    \end{algorithm}
求通过该三点的圆弧方程,利用MATLAB\footnote{笔者使用的版本为MATLAB R2012b}计算得到半径及圆心$r_c, x_c, y_c$,将圆心逆时针旋转$\theta=\pi/z$角度,得到最终圆心坐标
\begin{equation}
	\begin{pmatrix}
		x_t \\ y_t
	\end{pmatrix}
	=
	\begin{pmatrix}
		\cos \theta & \sin \theta \\
		-\sin \theta & \cos \theta
	\end{pmatrix}
	\begin{pmatrix}
		x_c \\ y_c
	\end{pmatrix}
\end{equation}
生成到Excel\footnote{笔者使用的版本为Excel 2013}表格中。\\
绘图结果如图\ref{fig2}所示\\
\begin{figure}[htb]
	\includegraphics[width=15cm]{zhou.png}
	\caption{轴系结构三维图}
	\label{fig2}
\end{figure}

\begin{thebibliography}{1}
\bibitem[1]{wang}Wang, Principle, Siemens Review, 1971, 34(5):217-219
\end{thebibliography}

\end{document}
